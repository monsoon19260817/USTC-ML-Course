%%%%%%%%%%%%%%%%%%%%%%%%%%%%%%%%%%%%%%%%%%%%%%%%%%%%%%%%%%%%%%%%
%
%  Template for homework of Introduction to Machine Learning.
%
%  Fill in your name, lecture number, lecture date and body
%  of homework as indicated below.
%
%%%%%%%%%%%%%%%%%%%%%%%%%%%%%%%%%%%%%%%%%%%%%%%%%%%%%%%%%%%%%%%%


\documentclass[11pt,letter,notitlepage]{article}
%Mise en page
\usepackage[left=2cm, right=2cm, lines=45, top=0.8in, bottom=0.7in]{geometry}
\usepackage{fancyhdr}
\usepackage{fancybox}
\usepackage{graphicx}
\usepackage{pdfpages} 
\usepackage{enumitem}
\usepackage{algorithm}
\usepackage{algorithmic}
\renewcommand{\headrulewidth}{1.5pt}
\renewcommand{\footrulewidth}{1.5pt}
\newcommand\Loadedframemethod{TikZ}
\usepackage[framemethod=\Loadedframemethod]{mdframed}

\usepackage{amssymb,amsmath}
\usepackage{amsthm}
\usepackage{thmtools}
\newtheorem{lemma}{Lemma}

\setlength{\topmargin}{0pt}
\setlength{\textheight}{9in}
\setlength{\headheight}{0pt}

\setlength{\oddsidemargin}{0.25in}
\setlength{\textwidth}{6in}

\usepackage{graphicx} % more modern
\usepackage{subfigure}
\usepackage{threeparttable}

%%%%%%%%%%%%%%%%%%%%%%%%
%%%%%% Define math operator %%%%%
%%%%%%%%%%%%%%%%%%%%%%%%
\DeclareMathOperator*{\argmin}{\bf argmin}
\DeclareMathOperator*{\relint}{\bf relint\,}
\DeclareMathOperator*{\dom}{\bf dom\,}
\DeclareMathOperator*{\intp}{\bf int\,}
%%%%%%%%%%%%%%%%%%%%%%%


\setlength{\topmargin}{0pt}
\setlength{\textheight}{9in}
\setlength{\headheight}{0pt}

\setlength{\oddsidemargin}{0.25in}
\setlength{\textwidth}{6in}
\pagestyle{fancy}
%%%%%%%%%%%%%%%%%%%%%%%%
%% Define the Exercise environment %%
%%%%%%%%%%%%%%%%%%%%%%%%
\mdtheorem[
topline=false,
rightline=false,
leftline=false,
bottomline=false,
leftmargin=-10,
rightmargin=-10
]{exercise}{\textbf{Exercise}}
%%%%%%%%%%%%%%%%%%%%%%%
%% End of the Exercise environment %%
%%%%%%%%%%%%%%%%%%%%%%%


%%%%%%%%%%%%%%%%%%%%%%%%
%% Define the Problem environment %%
%%%%%%%%%%%%%%%%%%%%%%%%
\mdtheorem[
topline=false,
rightline=false,
leftline=false,
bottomline=false,
leftmargin=-10,
rightmargin=-10
]{problem}{\textbf{Problem}}
%%%%%%%%%%%%%%%%%%%%%%%
%% End of the Exercise environment %%
%%%%%%%%%%%%%%%%%%%%%%%

%%%%%%%%%%%%%%%%%%%%%%%
%% Define the Solution Environment %%
%%%%%%%%%%%%%%%%%%%%%%%
\declaretheoremstyle
[
spaceabove=0pt, 
spacebelow=0pt, 
headfont=\normalfont\bfseries,
notefont=\mdseries, 
notebraces={(}{)}, 
headpunct={:\quad}, 
headindent={},
postheadspace={ }, 
postheadspace=4pt, 
bodyfont=\normalfont, 
qed=$\blacksquare$,
preheadhook={\begin{mdframed}[style=myframedstyle]},
	postfoothook=\end{mdframed},
]{mystyle}

\declaretheorem[style=mystyle,title=Solution,numbered=no]{solution}
\mdfdefinestyle{myframedstyle}{%
	topline=false,
	rightline=false,
	leftline=false,
	bottomline=false,
	skipabove=-6ex,
	leftmargin=-10,
	rightmargin=-10}
%%%%%%%%%%%%%%%%%%%%%%%
%% End of the Solution environment %%
%%%%%%%%%%%%%%%%%%%%%%%

%% Homework info.
\newcommand{\posted}{\text{Nov. 20, 2019}}       			%%% FILL IN POST DATE HERE
\newcommand{\due}{\text{Nov. 27, 2019}} 			%%% FILL IN Due DATE HERE
\newcommand{\hwno}{\text{5}} 		           			%%% FILL IN LECTURE NUMBER HERE


%%%%%%%%%%%%%%%%%%%%
%% Put your information here %%
%%%%%%%%%%%%%%%%%%%
\newcommand{\name}{\text{Jiahuan Yu}}  	          			%%% FILL IN YOUR NAME HERE
\newcommand{\id}{\text{PB17121687}}		       			%%% FILL IN YOUR ID HERE
%%%%%%%%%%%%%%%%%%%%
%% End of the student's info %%
%%%%%%%%%%%%%%%%%%%


\newcommand{\proj}[2]{\textbf{P}_{#2} (#1)}
\newcommand{\lspan}[1]{\textbf{span}  (#1)  }
\newcommand{\rank}[1]{ \textbf{rank}  (#1)  }
\newcommand{\RNum}[1]{\uppercase\expandafter{\romannumeral #1\relax}}


\lhead{
	\textbf{\name}
}
\rhead{
	\textbf{\id}
}
\chead{\textbf{
		Homework \hwno
}}


\begin{document}
\vspace*{-4\baselineskip}
\thispagestyle{empty}


\begin{center}
	{\bf\large Introduction to Machine Learning}\\
	{Fall 2019}\\
	University of Science and Technology of China
\end{center}

\noindent
Lecturer: Jie Wang  			 %%% FILL IN LECTURER HERE
\hfill
Homework \hwno
\\
Posted: \posted
\hfill
Due: \due
\\
Name: \name
\hfill
ID: \id
\hfill

\noindent
\rule{\textwidth}{2pt}

\medskip





%%%%%%%%%%%%%%%%%%%%%%%%%%%%%%%%%%%%%%%%%%%%%%%%%%%%%%%%%%%%%%%%
%% BODY OF HOMEWORK GOES HERE
%%%%%%%%%%%%%%%%%%%%%%%%%%%%%%%%%%%%%%%%%%%%%%%%%%%%%%%%%%%%%%%%

\textbf{Notice, }to get the full credits, please show your solutions step by step.

\begin{exercise}[Decision Tree  \textnormal{10pts}]
	Please build a decision tree based on the information gain to classify the following dataset (you need to show the calculation steps in detail).
\end{exercise}
\begin{table}[h]
	\centering
	\begin{tabular}{|c|c|c|c|c|c|}
		\hline
		Sample & $A_1$ & $A_2$ & $A_3$ & Response \\
		\hline
		$x_1$  & 1     & 0     & 0     & 0        \\
		$x_2$  & 1     & 0     & 1     & 0        \\
		$x_3$  & 0     & 1     & 0     & 0        \\
		$x_4$  & 1     & 1     & 1     & 1        \\
		$x_5$  & 1     & 1     & 0     & 1        \\
		\hline
	\end{tabular}
	\caption{Dataset} \label{tab:sample}
	{\raggedright The dataset consists of five samples $x_1,x_2,x_3,x_4,x_5$. For each sample, we can observe the features $A_1,A_2,A_3$ and the corresponding response. \par}
\end{table}
\begin{solution}

\end{solution}
\newpage

\begin{exercise}[Softmax and Cross Entropy  \textnormal{30pts}]
	The softmax function $f:\mathbb{R}^n\rightarrow\mathbb{R}^n$ is defined by:
	$$f_i(x)=\frac{\exp(x_i)}{\sum_{k=1}^{n}\exp(x_k)}, i=1,\ldots,n,$$
	where $x_i$ is the $i^{th}$ component of $x\in\mathbb{R}^n$. The function  $f(x)=(f_1(x),f_2(x),\ldots,f_n(x))^{\top}$ converts each input $x$ into a probability (stochastic) vector in which all entries are nonnegative and add up to one.
	\begin{itemize}
		\item[1.] Please find the gradient and Jacobian of $f(x)$, i.e., $\nabla f(x)$ and $Df(x)$.

		\item[2.] Show that $f(x)=f(x-c)$, where $c=\max\{x_1,x_2,...,x_n\}$. When could we need this transformation?

		\item[3.] Please find the gradient of cross entropy function:
		      $$g(x)=-\sum_{i=1}^{n}H_i\log(f_i(x)),$$
		      where $H\in\mathbb{R}^n$ is a one-hot vector.
	\end{itemize}
\end{exercise}

\begin{solution}

\end{solution}

\newpage


\begin{exercise}[Convolutional Neural Network \textnormal{40pts}]
	\begin{enumerate}
		\item The average pooling in convolutional neural network can be formulated as
		      $$f_1(x)= \frac{\sum_{i=1}^n x_i}{n},$$
		      where $x_i$ is the $i^{th}$ component of $x\in\mathbb{R}^n$. Please derive the gradient of $f_1(x)$.
		\item The max pooling in convolutional neural network can be formulated as
		      $$f_2(x)= \max\{x_1,\dots,x_n\},$$
		      where $x_i$ is the $i^{th}$ component of $x\in\mathbb{R}^n$.
		      \begin{enumerate}
			      \item Find the set containing all differentiable points of of $f_2$.
			      \item We call $d(x)$ is a subgradient at $x$ $f_2$ if
			            \begin{align*}
				            f_2(y) \geq f_2(x) + \langle d(x),y-x \rangle, \forall x,y .
			            \end{align*}
			            Find a subgradient $d(x)$ of $f_2$ at $x$.
		      \end{enumerate}
		\item Suppose that we have a convolutional neural network as shown in Table \ref{tab:cnn}.
		      \begin{enumerate}
			      \item The convolutinal layer parameters are denoted as ``conv$\langle$filter size$\rangle$-$\langle$number of filters$\rangle$".
			      \item  The fully connected layer parameters are denoted as ``FC$\langle$number of neurons$\rangle$".
			      \item The window size of pooling layers is $3$.
			      \item The stride of convolutinal layers is $1$.
			      \item The stride of pooling layers is $3$.
			      \item There is no padding in both convolutional and pooling layers.
			      \item For convenience, we assume that there is no activation function and bias.
		      \end{enumerate}

		      Suppose that the input is a $\mathbf{386\times 386}$ \textbf{RGB} image. Please derive the size of all feature maps and the number of parameters.
	\end{enumerate}


\end{exercise}

\begin{table}[h]
	\centering
	\begin{tabular}{|c|c|c|c|c|c|c|}
		\hline
		conv3-64 & max pool & conv3-256 & conv1-512 & max pool & FC-2048 & FC-1000 \\
		\hline
	\end{tabular}
	\caption{The architecture of convolutional neural network} \label{tab:cnn}
\end{table}
\begin{solution}

\end{solution}

\newpage

\begin{exercise}[Matrix Calculus \textnormal{20pts}]

	\indent Let $L = f(h(Ax+b))$, where $A \in \mathbb{R}^{m \times n}$, $x \in \mathbb{R}^n$, $b \in \mathbb{R}^m$, and $f: \mathbb{R}^m \rightarrow \mathbb{R}$. Define $z =  Ax + b \in \mathbb{R}^m$ and $w =  h(z)  = ( \sigma(z_1),\ldots,\sigma(z_n) )^{\top}$, where $z_i$ is the $i^{th}$ component of $z$ and
	\begin{align*}
		\sigma(z_i) = \frac{1}{1+\exp(-z_i)}.
	\end{align*}
	Assume $\nabla_w f$ is known.
	\begin{enumerate}
		\item Please derive $\nabla_{x}L$.
		\item Please derive
		      \begin{align*}
			      \nabla_{A}L = \left[
				      \begin{matrix}
					       & \frac{\partial L}{\partial A_{11}} & \, \dots \, & \frac{\partial L}{\partial A_{1j}} & \, \dots \, & \frac{\partial L}{\partial A_{1n}} \\
					       & \dots                              & \, \dots \, & \dots                              & \, \dots \, & \dots                              \\
					       & \frac{\partial L}{\partial A_{i1}} & \, \dots \, & \frac{\partial L}{\partial A_{ij}} & \, \dots \, & \frac{\partial L}{\partial A_{in}} \\
					       & \dots                              & \, \dots \, & \dots                              & \, \dots \, & \dots                              \\
					       & \frac{\partial L}{\partial A_{m1}} & \, \dots \, & \frac{\partial L}{\partial A_{mj}} & \, \dots \, & \frac{\partial L}{\partial A_{mn}}
				      \end{matrix}
				      \right],
		      \end{align*}
		      where $A_{i,j}$ is the entry in the $i^{th}$ row, $j^{th}$ column of the matrix $A$.
	\end{enumerate}

\end{exercise}
\begin{solution}

\end{solution}



%%%%%%%%%%%%%%%%%%%%%%%%%%%%%%%%%%%%%%%%%%%%%%%%%%%%%%%%%%%%%%%%

\end{document}
